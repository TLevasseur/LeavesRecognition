\subsection{Cas simple : deux espèces - Méthode de la séparation à vaste marge}
\subsubsection{Protocole}
Dans un premier temps nous avons cherché à résoudre le problème de façon simple, c'est à dire ne travailler que sur deux classes.
Suite à la définition de deux groupes de feuilles (ortie et érable dans notre exemple), il était possible de définir un hyperplan séparant ces deux groupes de feuilles grâce à la méthode de séparation à vaste marge.


\paragraph{}
Dans un premier temps, les feuilles sont étiquetées suivant leur classe par les valeurs \{ -1 ; 1 \}. 
\paragraph{}
On cherche ensuite une fonction telle que :
\smallbreak
 $\forall$x$\longrightarrow$f(x)$\in$\{ -1; 1 \}

\paragraph{}
L'identification de la feuille comme étant d'une espèce ou de l'autre correspond au résultat obtenu grâce à cette fonction.

\subsubsection{Résultats}
Les résultats obtenus lors de cette première série de test sont concluants dans la mesure où 100 \% des tests ont été fructueux. Cependant la petite taille de l'échantillon ne permet pas de réellement conclure sur la fiabilité des résultats même si ceux-ci sont plus que prometteurs.