\documentclass{article}
\usepackage[utf8]{inputenc}
\usepackage[cyr]{aeguill}
\usepackage[francais]{babel}

\title{Reconnaissance des espèces de plantes grâce à la forme de leurs feuilles}

\author{Marie Lavigne, Thibaud Levasseur, Nicolas Monet}

\begin{document}
\maketitle
\newpage
\tableofcontents
\newpage
\section{Introduction}
On s'inspire du travail des trois chercheurs \\
On va utiliser leurs recherches pour déterminer comment classer les feuilles, générer des classes de feuille, et quels critères calculer pour décider de la classe d'une feuille.
\newpage
\section{Développement}
Le traitement de l'image est très basique: on charge une image en noir et blanc, on en trace le contour, et on calcule des paramètres propres à l'image (zones d'intéret, aire, périmètre...)
\smallbreak
Ce programme est basé sur l'apprentissage des formes, et sur du clustering. \\
On a une banque de données, des images de feuilles, qu'on traite pour calculer les paramètres décrits précédemment, et on labelle ces données: telle feuille a telles caractéristiques et c'est une feuille d'Erable, une autre a telles caractéristiques et c'est une feuille d'Ortie. \\
A partir de cet apprentissage on obtient des classes.
\smallbreak
Ensuite, on calcule les caractéristiques de la feuille à analyser. Par la méthode des KPPV (Plus Proches Voisins) on cherche la classe la plus proche de notre feuille, qui est la classe la plus ressemblante. \\
On détermine ainsi de quel type de feuille il s'agit.
\newpage
\section{Conclusion}
Avec nos données de test (16 et 12 feuilles de 2 types différents) on obtient 100 \% de réussite.\\
Le temps d'apprentissage est d'environ une seconde par image à traiter. \\
Le temps d'exécution est d'environ une seconde, qui est utilisé très majoritairement par la détermination des caractéristiques de la feuille. L'algorithme KPPV est relativement rapide et son temps d'exécution est négligeable en comparaison.
\newpage
\section{Annexes}
Concernant le code: le programme utilise une fonction TIMVecteur qui renvoie un vecteur avec les caractéristiques de l'image.

\end{document}
