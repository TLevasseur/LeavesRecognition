\documentclass{article}
\usepackage[utf8]{inputenc}
\usepackage[cyr]{aeguill}
\usepackage[francais]{babel}
\usepackage{graphicx}
\usepackage{amsmath}


\title{Reconnaissance des espèces de plantes grâce à la forme de leurs feuilles}

\author{Marie Lavigne, Thibaud Levasseur, Nicolas Monet}

\begin{document}
\maketitle
\newpage
\tableofcontents
\newpage
\section{Introduction}
La reconnaissance des plantes � toujours �t� un enjeu pour les populations, tant pour leurs suries et la reconnaissance des plantes comestibles, que la protection de l'environnement.
Il apparait cependant que reconnaitre une plante ou un arbre n'est pas forc�ment facile pour toute personne non-aguerrie.
Il s'av�re donc n�cessaire de cr�er un outil d'aide � la d�cision qui permettrait de classer les esp�ces gr�ce � une photo par exemple.
Il est donc n�cessaire de trouver une caract�ristique distincte pour toutes les esp�ces de plantes qui soit disponible tout au long de l'ann�e.
Les fleurs n'�tant disponibles que pendant une p�riode �ph�m�re (en plus d'�tre en 3 dimensions), les deux alternatives possibles sont l'�tude de l'�corce ou des feuilles. Nous nous pencherons sur les feuilles afin de pouvoir aussi int�grer les fleurs � cette �tude.
\paragraph{}
Ce projet a �t� r�alis� gr�ce au travail des chercheurs Ji-Xiang Du, Xiao-Feng Wang et Guo-Jun Zang publi� sur science-direct.com ( 185 (2007) 883?893). Les donn�es utilis�es sont issues du site internet http://archive.ics.uci.edu/ml/datasets.html.
\newpage

\section{Description des données}
Le jeu de donné utilisé comporte 360 feuilles en images couleur et noir et blanc réparties et triées en 36 espèces. Le nombres d'individu par groupe est assez réduit et ne dépasse pas le nombre de 16. Une imprécision peut donc apparaitre dû au nombre réduit d'individu dans chaque échantillon.
\paragraph{}
Les images ont une résolution de 960*720 pixel, ce qui offre précision suffisante, sans être excessive. Cependant le temps de calcul pour la sérialisation des données reste assez long : quelques secondes par image. Une sauvegarde de celle-ci est donc fortement conseillée.
\paragraph{}
La représentation des feuilles est homogène : l'image est centrée et pour chaque feuille le pétiole est placé vers le bas de l'image, ce qui permet de faciliter grandement le traitement de l'image dans la mesure ou aucune rotation n'est nécessaire en pré-traitement.

\begin{center}
	\includegraphics[scale=0.11]{image/urticaRGB.png}	\includegraphics[scale=0.15]{image/urticaNB.png}
\end{center}
\section{Traitement des données}
\subsection{Définitions des propriétés géométriques invariantes}


Afin de séparer nos données, il est nécessaire de trouver les propriétés géométriques invariantes de chaque espèces de feuilles qui serviront à définir une hypersphères pour chaque espèce.

Afin de ne pas avoir de calculs trop lourds, nous avons faite le choix de nous limiter à une caractérisation de chaque feuille par un vecteur x $\in \Re ^{5}$. À ce vecteur sera ajouté un label pour l'apprentissage.

\subsubsection{Rectangularité}
La réctangularité est définie comme le rapport de la longueur sur la largeur du plus petit rectangle entourant la feuille.
\smallbreak
$R=\frac{longueur}{largeur}$
\subsubsection{Sphéricité}
La sphéricité est définie comme le rapport entre le rayon du cercle inscrit sur le rayon du cercle circonscrit.
\smallbreak
$S=\frac{r_{i}}{r_{c}}$
\subsubsection{Ratio d'aire}
Le ratio d'aire est le rapport entre l'air du polygone concave le plus petit circonscrit à la feuille et celle le plus petit rectangle entourant la feuille.
\smallbreak
$AO=\frac{A_{r}}{A_{C}}$
\subsubsection{Ratio de périmètre}
Le ratio de périmètre est le rapport entre le périmètre du polygone concave le plus petit circonscrit à la feuille et celui du plus petit rectangle entourant la feuille.
\smallbreak
$CP=\frac{P_{r}}{P_{C}}$
\subsubsection{Facteur de forme}
Le facteur de forme est une valeur utilisé pour décrire une feuille qui semble être empirique comme efficace. Elle est définit comme suit.
\smallbreak
$F=\frac{4*\pi*R_{r}}{P_{r}^{2}}$
\subsection{Cas simple : deux espèces - Méthode de la séparation à vaste marge}
\subsubsection{Protocole}
Dans un premier temps nous avons cherché à résoudre le problème de façon simple, c'est à dire ne travailler que sur deux classes.
Suite à la définition de deux groupes de feuilles (ortie et érable dans notre exemple), il était possible de définir un hyperplan séparant ces deux groupes de feuilles grâce à la méthode de séparation à vaste marge.


\paragraph{}
Dans un premier temps, les feuilles sont étiquetées suivant leur classe par les valeurs \{ -1 ; 1 \}. 
\paragraph{}
On cherche ensuite une fonction telle que :
\smallbreak
 $\forall$x$\longrightarrow$f(x)$\in$\{ -1; 1 \}

\paragraph{}
L'identification de la feuille comme étant d'une espèce ou de l'autre correspond au résultat obtenu grâce à cette fonction.

\subsubsection{Résultats}
Les résultats obtenus lors de cette première série de test sont concluants dans la mesure où 100 \% des tests ont été fructueux. Cependant la petite taille de l'échantillon ne permet pas de réellement conclure sur la fiabilité des résultats même si ceux-ci sont plus que prometteurs.

\newpage
\section{Conclusion}
Avec nos données de test (16 et 12 feuilles de 2 types différents) nous obtenons un taux de réussite de 100 \%.\\
Le temps d'apprentissage pour chaque image traité est d'environ une seconde. \\
Le temps d'exécution, quant à lui est autour d'une seconde, temps utilisé très majoritairement par la détermination des caractéristiques de la feuille. 
En conclusion nous pouvons dire que l'algorithme KPPV est rapide (et pourrait même l'être encore plus avec un appareil plus performant) mais aussi très efficace dans notre cas de test.
Il est cependant important de relativiser cette efficacité puisque nous avons sciemment choisi des feuilles de formes peu semblables. Il serait donc intéressant dans une extension de ce projet, de voir si des feuilles d'espèces de plantes différentes mais ayant des formes similaires auraient un taux de reconnaissance aussi élevé et si tel n'était pas le cas, quelles améliorations apportées.
\newpage
\section{Annexes}
Concernant le code: le programme utilise une fonction TIMVecteur qui renvoie un vecteur avec les caractéristiques de l'image.

\end{document}
