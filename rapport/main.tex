\documentclass{article}
\usepackage[utf8]{inputenc}
\usepackage[cyr]{aeguill}
\usepackage[francais]{babel}

\title{Reconnaissance des espèces de plantes grâce à la forme de leurs feuilles}

\author{Marie Lavigne, Thibaud Levasseur, Nicolas Monet}

\begin{document}
\maketitle
\newpage
\tableofcontents
\newpage
\section{Introduction}
La reconnaissance des plantes � toujours �t� un enjeu pour les populations, tant pour leurs suries et la reconnaissance des plantes comestibles, que la protection de l'environnement.
Il apparait cependant que reconnaitre une plante ou un arbre n'est pas forc�ment facile pour toute personne non-aguerrie.
Il s'av�re donc n�cessaire de cr�er un outil d'aide � la d�cision qui permettrait de classer les esp�ces gr�ce � une photo par exemple.
Il est donc n�cessaire de trouver une caract�ristique distincte pour toutes les esp�ces de plantes qui soit disponible tout au long de l'ann�e.
Les fleurs n'�tant disponibles que pendant une p�riode �ph�m�re (en plus d'�tre en 3 dimensions), les deux alternatives possibles sont l'�tude de l'�corce ou des feuilles. Nous nous pencherons sur les feuilles afin de pouvoir aussi int�grer les fleurs � cette �tude.
\paragraph{}
Ce projet a �t� r�alis� gr�ce au travail des chercheurs Ji-Xiang Du, Xiao-Feng Wang et Guo-Jun Zang publi� sur science-direct.com ( 185 (2007) 883?893). Les donn�es utilis�es sont issues du site internet http://archive.ics.uci.edu/ml/datasets.html.
\newpage
\section{Développement}
Le traitement de l'image est très basique: on charge une image en noir et blanc, on en trace le contour, et on calcule des paramètres propres à l'image (zones d'intéret, aire, périmètre...)
\smallbreak
Ce programme est basé sur l'apprentissage des formes, et sur du clustering. \\
On a une banque de données, des images de feuilles, qu'on traite pour calculer les paramètres décrits précédemment, et on labelle ces données: telle feuille a telles caractéristiques et c'est une feuille d'Erable, une autre a telles caractéristiques et c'est une feuille d'Ortie. \\
A partir de cet apprentissage on obtient des classes.
\smallbreak
Ensuite, on calcule les caractéristiques de la feuille à analyser. Par la méthode des KPPV (Plus Proches Voisins) on cherche la classe la plus proche de notre feuille, qui est la classe la plus ressemblante. \\
On détermine ainsi de quel type de feuille il s'agit.
\newpage
\section{Conclusion}
Avec nos données de test (16 et 12 feuilles de 2 types différents) on obtient 100 \% de réussite.\\
Le temps d'apprentissage est d'environ une seconde par image à traiter. \\
Le temps d'exécution est d'environ une seconde, qui est utilisé très majoritairement par la détermination des caractéristiques de la feuille. L'algorithme KPPV est relativement rapide et son temps d'exécution est négligeable en comparaison.
\newpage
\section{Annexes}
Concernant le code: le programme utilise une fonction TIMVecteur qui renvoie un vecteur avec les caractéristiques de l'image.

\end{document}
