La reconnaissance des plantes à toujours été un enjeu pour les populations, tant pour leurs suries et la reconnaissance des plantes comestibles, que la protection de l'environnement.
Il apparait cependant que reconnaitre une plante ou un arbre n'est pas forcément facile pour toute personne non-aguerrie.
Il s'avère donc nécessaire de créer un outil d'aide à la décision qui permettrait de classer les espèces grâce à une photo par exemple.
Il est donc nécessaire de trouver une caractéristique distincte pour toutes les espèces de plantes qui soit disponible tout au long de l'année.
Les fleurs n'étant disponibles que pendant une période éphémère (en plus d'être en 3 dimensions), les deux alternatives possibles sont l'étude de l'écorce ou des feuilles. Nous nous pencherons sur les feuilles afin de pouvoir aussi intégrer les fleurs à cette étude.
\paragraph{}
Ce projet a été réalisé grâce au travail des chercheurs Ji-Xiang Du, Xiao-Feng Wang et Guo-Jun Zang publié sur science-direct.com ( 185 (2007) 883–93). Les données utilisées sont issues du site internet http://archive.ics.uci.edu/ml/datasets.html.