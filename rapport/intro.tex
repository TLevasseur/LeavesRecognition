La reconnaissance des plantes � toujours �t� un enjeu pour les populations, tant pour leurs suries et la reconnaissance des plantes comestibles, que la protection de l'environnement.
Il apparait cependant que reconnaitre une plante ou un arbre n'est pas forc�ment facile pour toute personne non-aguerrie.
Il s'av�re donc n�cessaire de cr�er un outil d'aide � la d�cision qui permettrait de classer les esp�ces gr�ce � une photo par exemple.
Il est donc n�cessaire de trouver une caract�ristique distincte pour toutes les esp�ces de plantes qui soit disponible tout au long de l'ann�e.
Les fleurs n'�tant disponibles que pendant une p�riode �ph�m�re (en plus d'�tre en 3 dimensions), les deux alternatives possibles sont l'�tude de l'�corce ou des feuilles. Nous nous pencherons sur les feuilles afin de pouvoir aussi int�grer les fleurs � cette �tude.
\paragraph{}
Ce projet a �t� r�alis� gr�ce au travail des chercheurs Ji-Xiang Du, Xiao-Feng Wang et Guo-Jun Zang publi� sur science-direct.com ( 185 (2007) 883?893). Les donn�es utilis�es sont issues du site internet http://archive.ics.uci.edu/ml/datasets.html.