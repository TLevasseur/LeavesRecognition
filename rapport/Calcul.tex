\subsection{Définitions des propriétés géométriques invariantes}


Afin de séparer nos données, il est nécessaire de trouver les propriétés géométriques invariantes de chaque espèces de feuilles qui serviront à définir une hypersphères pour chaque espèce.

Afin de ne pas avoir de calculs trop lourds, nous avons faite le choix de nous limiter à une caractérisation de chaque feuille par un vecteur x $\in \Re ^{5}$. À ce vecteur sera ajouté un label pour l'apprentissage.

\subsubsection{Rectangularité}
La réctangularité est définie comme le rapport de la longueur sur la largeur du plus petit rectangle entourant la feuille.
\smallbreak
$R=\frac{longueur}{largeur}$
\subsubsection{Sphéricité}
La sphéricité est définie comme le rapport entre le rayon du cercle inscrit sur le rayon du cercle circonscrit.
\smallbreak
$S=\frac{r_{i}}{r_{c}}$
\subsubsection{Ratio d'aire}
Le ratio d'aire est le rapport entre l'air du polygone concave le plus petit circonscrit à la feuille et celle le plus petit rectangle entourant la feuille.
\smallbreak
$AO=\frac{A_{r}}{A_{C}}$
\subsubsection{Ratio de périmètre}
Le ratio de périmètre est le rapport entre le périmètre du polygone concave le plus petit circonscrit à la feuille et celui du plus petit rectangle entourant la feuille.
\smallbreak
$CP=\frac{P_{r}}{P_{C}}$
\subsubsection{Facteur de forme}
Le facteur de forme est une valeur utilisé pour décrire une feuille qui semble être empirique comme efficace. Elle est définit comme suit.
\smallbreak
$F=\frac{4*\pi*R_{r}}{P_{r}^{2}}$